\documentclass[listings]{labreport}
\usepackage{amsmath}
\subject{Теория автоматов}
\titleparts{Практическое задание №3}{Вариант 8}
\students{Лабушев Тимофей}
\usepackage{caption}
\captionsetup{justification=raggedright, singlelinecheck=false}

\begin{document}

\maketitlepage

\section*{Цель работы}

Практическое освоение метода перехода от абстрактного автомата
к структурному автомату.

\section*{Задание}

Абстрактный автомат задан табличным способом. Причем абстрактный автомат Мили
представлен таблицами переходов и выходов, а абстрактный автомат Мура — одной
отмеченной таблицей переходов. Для синтеза структурного автомата использовать
функционально полную систему логических элементов И, ИЛИ, НЕ и автомат Мура,
обладающий полнотой переходов и полнотой выходов. Синтезированный структурный
автомат представить в виде ПАМЯТИ и КОМБИНАЦИОННОЙ СХЕМЫ.

\section*{Исходный автомат Мили}

\begin{tabular}{|*{7}{c|}}
\hline
$\delta$ & $a_1$ & $a_2$ & $a_3$ & $a_4$ & $a_5$ & $a_6$\\\hline
$z_1$ & $a_2$ & $a_1$ & $a_5$ & $a_6$ & $a_2$ & $a_3$\\\hline
$z_2$ & $a_4$ & $a_3$ & $a_1$ & $a_4$ & $a_6$ & $a_5$\\\hline
$z_3$ & $a_6$ & $a_5$ & $a_3$ & $a_2$ & $a_4$ & $a_1$\\\hline
\end{tabular}
\begin{tabular}{|*{7}{c|}}
\hline
$\lambda$ & $a_1$ & $a_2$ & $a_3$ & $a_4$ & $a_5$ & $a_6$\\\hline
$z_1$ & $w_1$ & $w_2$ & $w_3$ & $w_3$ & $w_2$ & $w_1$\\\hline
$z_2$ & $w_2$ & $w_3$ & $w_1$ & $w_2$ & $w_1$ & $w_3$\\\hline
$z_3$ & $w_3$ & $w_1$ & $w_2$ & $w_1$ & $w_3$ & $w_2$\\\hline
\end{tabular}

\section*{Двоичное кодирование исходного автомата}

Входной алфавит:

\begin{tabular}{|*{3}{c|}}
\hline
$$ & $x_1$ & $x_2$\\\hline
$z_1$ & $0$ & $0$\\\hline
$z_2$ & $0$ & $1$\\\hline
$z_3$ & $1$ & $0$\\\hline
\end{tabular}

Выходной алфавит:

\begin{tabular}{|*{3}{c|}}
\hline
$$ & $y_1$ & $y_2$\\\hline
$w_1$ & $0$ & $0$\\\hline
$w_2$ & $0$ & $1$\\\hline
$w_3$ & $1$ & $0$\\\hline
\end{tabular}

Состояния:

\begin{tabular}{|*{4}{c|}}
\hline
$$ & $Q_1$ & $Q_2$ & $Q_3$\\\hline
$a_1$ & $0$ & $0$ & $0$\\\hline
$a_2$ & $0$ & $0$ & $1$\\\hline
$a_3$ & $0$ & $1$ & $0$\\\hline
$a_4$ & $0$ & $1$ & $1$\\\hline
$a_5$ & $1$ & $0$ & $0$\\\hline
$a_6$ & $1$ & $0$ & $1$\\\hline
\end{tabular}

\section*{Таблицы переходов и выходов структурного автомата}

\begin{tabular}{|*{7}{c|}}
\hline
$x_1x_2/Q_1Q_2Q_3$ & 000 & 001 & 010 & 011 & 100 & 101\\\hline
00 & 001 & 000 & 100 & 101 & 001 & 010\\\hline
01 & 011 & 010 & 000 & 011 & 101 & 100\\\hline
10 & 101 & 100 & 010 & 001 & 011 & 000\\\hline
\end{tabular}

\begin{tabular}{|*{7}{c|}}
\hline
$x_1x_2/Q_1Q_2Q_3$ & 000 & 001 & 010 & 011 & 100 & 101\\\hline
00 & 00 & 01 & 10 & 10 & 01 & 00\\\hline
01 & 01 & 10 & 00 & 01 & 00 & 10\\\hline
10 & 10 & 00 & 01 & 00 & 10 & 01\\\hline
 & $y_1y_2$ & $y_1y_2$ & $y_1y_2$ & $y_1y_2$ & $y_1y_2$ & $y_1y_2$\\\hline
\end{tabular}

\end{document}
