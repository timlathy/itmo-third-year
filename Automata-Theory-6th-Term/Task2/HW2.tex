\documentclass[listings]{labreport}
\usepackage{amsmath}
\subject{Теория автоматов}
\titleparts{Практическое задание №2}{Вариант 8}
\students{Лабушев Тимофей}
\usepackage{caption}
\captionsetup{justification=raggedright, singlelinecheck=false}

\begin{document}

\maketitlepage

\section*{Цель работы}

Овладение навыками минимизации полностью определенных абстрактных автоматов
(на примере автомата Мура).

\section*{Задание}

\begin{enumerate}
\item В соответствии с номером варианта выбрать абстрактный автомат $S=(A,Z,W,\delta,\lambda,a_1)$.
\item Найти последовательные разбиения $\pi_1,\pi_2,\ldots,\pi_k,\pi_{k+1}$ множества $А$ на классы одно-, двух-, \ldots, $k+1$ - эквивалентных между собой состояний.
\item Разбиение на классы производить до тех пор, пока на каком-то $k+1$ шаге не окажется, что $\pi k+1 = \pi k$.
\item В каждом классе эквивалентности разбиения $\pi$ выбрать по одному элементу, которые образуют множество $А'$ состояний минимального автомата $S'=(A',Z,W,\delta',\lambda',a_1')$, эквивалентного исходному автомату $S$.
\item Функции переходов и выходов автомата $S'$ определить на множестве $A' *Z$, то есть $\delta': A' *Z \mapsto A', \lambda': A' *Z \mapsto W$.
\item В качестве $a_1'$ выбрать одно из состояний, эквивалентных $a_1$.
\item Используя навыки полученные при выполнении практического задания 1, осуществить проверку исходного и минимизированного автоматов на эквивалентность. 
\end{enumerate}

\section*{Исходный автомат Мура}

\begin{tabular}{|*{9}{c|}}
\hline
$\lambda$ & $w_2$ & $w_2$ & $w_2$ & $w_2$ & $w_1$ & $w_2$ & $w_2$ & $w_2$ \\\hline
$\delta$  & $a_1$ & $a_2$ & $a_3$ & $a_4$ & $a_5$ & $a_6$ & $a_7$ & $a_8$ \\\hline
$z_1$     & $a_1$ & $a_6$ & $a_1$ & $a_7$ & $a_1$ & $a_8$ & $a_3$ & $a_1$ \\\hline
$z_2$     & $a_4$ & $a_5$ & $a_2$ & $a_6$ & $a_3$ & $a_4$ & $a_1$ & $a_4$ \\\hline
\end{tabular}

\section*{Ход работы}

Найдем классы одноэквивалентных состояний по выходам:

$B_1 = \{a_5\}, B_2 = \{a_1, a_2, a_3, a_4, a_6, a_7, a_8\}$

Получим разбиение $\Pi_1 = \{B_1, B_2\}$.

\newpage
Заменим состояния в таблице переходов соответствующими классами эквивалентности:

\begin{tabular}{|*{9}{c|}}
\hline
      & $B_1$ & \multicolumn{7}{c|}{$B_2$}\\\hline
      & $a_5$ & $a_1$ & $a_2$ & $a_3$ & $a_4$ & $a_6$ & $a_7$ & $a_8$\\\hline
$z_1$ & $B_2$ & $B_2$ & $B_2$ & $B_2$ & $B_2$ & $B_2$ & $B_2$ & $B_2$\\\hline
$z_2$ & $B_2$ & $B_2$ & $B_1$ & $B_2$ & $B_2$ & $B_2$ & $B_2$ & $B_2$\\\hline
\end{tabular}

Найдем классы $k$-эквивалентных ($k=2$) состояний:

$C_1 = \{a_5\}, C_2 = \{a_2\}, C_3 = \{a_1, a_3, a_4, a_6, a_7, a_8\}$

Получим разбиение $\Pi_2 = \{C_1, C_2, C_3\}$, которое не совпадает с предыдущим.

Заменим состояния в таблице переходов соответствующими классами эквивалентности:

\begin{tabular}{|*{9}{c|}}
\hline
      & $C_1$ & $C_2$ & \multicolumn{6}{c|}{$C_3$}\\\hline
      & $a_5$ & $a_2$ & $a_1$ & $a_3$ & $a_4$ & $a_6$ & $a_7$ & $a_8$\\\hline
$z_1$ & $C_3$ & $C_3$ & $C_3$ & $C_3$ & $C_3$ & $C_3$ & $C_3$ & $C_3$\\\hline
$z_2$ & $C_3$ & $C_1$ & $C_3$ & $C_2$ & $C_3$ & $C_3$ & $C_3$ & $C_3$\\\hline
\end{tabular}

Найдем классы $k$-эквивалентных ($k=3$) состояний:

$D_1 = \{a_5\}, D_2 = \{a_2\}, D_3 = \{a_3\}, D_4 = \{a_1, a_4, a_6, a_7, a_8\}$

Заменим состояния в таблице переходов соответствующими классами эквивалентности:

\begin{tabular}{|*{9}{c|}}
\hline
      & $D_1$ & $D_2$ & $D_3$ & \multicolumn{5}{c|}{$D_4$}\\\hline
      & $a_5$ & $a_2$ & $a_3$ & $a_1$ & $a_4$ & $a_6$ & $a_7$ & $a_8$\\\hline
$z_1$ & $D_4$ & $D_4$ & $D_4$ & $D_4$ & $D_4$ & $D_4$ & $D_3$ & $D_4$\\\hline
$z_2$ & $D_3$ & $D_1$ & $D_2$ & $D_4$ & $D_4$ & $D_4$ & $D_4$ & $D_4$\\\hline
\end{tabular}

Найдем классы $k$-эквивалентных ($k=4$) состояний:

$E_1 = \{a_5\}, E_2 = \{a_2\}, E_3 = \{a_3\}, E_4 = \{a_7\}, E_5 = \{a_1, a_4, a_6, a_8\}$

Заменим состояния в таблице переходов соответствующими классами эквивалентности:

\begin{tabular}{|*{9}{c|}}
\hline
      & $E_1$ & $E_2$ & $E_3$ & $E_4$ & \multicolumn{4}{c|}{$E_5$}\\\hline
      & $a_5$ & $a_2$ & $a_3$ & $a_7$ & $a_1$ & $a_4$ & $a_6$ & $a_8$\\\hline
$z_1$ & $E_5$ & $E_5$ & $E_5$ & $E_3$ & $E_5$ & $E_4$ & $E_5$ & $E_5$\\\hline
$z_2$ & $E_3$ & $E_1$ & $E_2$ & $E_5$ & $E_5$ & $E_5$ & $E_5$ & $E_5$\\\hline
\end{tabular}

Найдем классы $k$-эквивалентных ($k=5$) состояний:

$F_1 = \{a_5\}, F_2 = \{a_2\}, F_3 = \{a_3\}, F_4 = \{a_7\}, F_5 = \{a_4\}, F_6 = \{a_1, a_6, a_8\}$

Заменим состояния в таблице переходов соответствующими классами эквивалентности:

\begin{tabular}{|*{9}{c|}}
\hline
      & $F_1$ & $F_2$ & $F_3$ & $F_4$ & $F_5$ & \multicolumn{3}{c|}{$F_6$}\\\hline
      & $a_5$ & $a_2$ & $a_3$ & $a_7$ & $a_4$ & $a_1$ & $a_6$ & $a_8$\\\hline
$z_1$ & $F_6$ & $F_6$ & $F_6$ & $F_3$ & $F_4$ & $F_6$ & $F_6$ & $F_6$\\\hline
$z_2$ & $F_3$ & $F_1$ & $F_2$ & $F_6$ & $F_6$ & $F_5$ & $F_5$ & $F_5$\\\hline
\end{tabular}

Найдем классы $k$-эквивалентных ($k=6$) состояний:

Получим разбиение $\Pi_6 = \{G_1, G_2, G_3, G_4, G_5, G_6\}$, которое
совпадает с предыдущим.

Минимизация завершена.

\subsection*{Полученный автомат}

\begin{tabular}{|*{9}{c|}}
\hline
$\lambda$ & $w_2$ & $w_2$ & $w_2$ & $w_2$ & $w_1$ & $w_2$  \\\hline
$\delta$  & $a_1$ & $a_2$ & $a_3$ & $a_4$ & $a_5$ & $a_6$ \\\hline
$z_1$     & $a_1$ & $a_1$ & $a_1$ & $a_6$ & $a_1$ & $a_3$ \\\hline
$z_2$     & $a_4$ & $a_5$ & $a_2$ & $a_1$ & $a_3$ & $a_1$ \\\hline
\end{tabular}

\subsection*{Проверка на эквивалентность}

Проверим исходный и минимизированный автоматы на эквивалентность,
используя входное слово, достаточное для осуществления всех возможных
переходов в исходном графе:

$z_1z_2z_1z_1z_2z_2z_2z_2z_1z_2z_2z_1z_2z_2z_1z_1z_2z_1z_1z_2z_1z_1z_1z_2z_1z_1z_2z_2z_1$

Реакция исходного автомата ($a_0 = a_1$):

$a_1a_4a_7a_3a_2a_5a_3a_2a_6a_4a_6a_8a_4a_6a_8a_1a_4a_7a_1a_4a_7a_3a_1a_4a_7a_3a_2a_5a_1$

$w_2w_2w_2w_2w_2w_1w_2w_2w_2w_2w_2w_2w_2w_2w_2w_2w_2w_2w_2w_2w_2w_2w_2w_2w_2w_2w_2w_1w_2$

Реакция минимизированного автомата ($a_0 = a_1$):

$a_1a_4a_6a_3a_2a_5a_3a_2a_1a_4a_1a_1a_4a_1a_1a_1a_4a_6a_3a_2a_1a_1a_1a_4a_6a_3a_2a_5a_1$

$w_2w_2w_2w_2w_2w_1w_2w_2w_2w_2w_2w_2w_2w_2w_2w_2w_2w_2w_2w_2w_2w_2w_2w_2w_2w_2w_2w_1w_2$

Выходные слова идентичны, автоматы эквивалентны.

\section*{Вывод}

В ходе выполнения работы был освоен навык минимизации абстрактного автомата при помощи
алгоритма минимизации, предложенного Ауфенкампом и Хоном, на примере автомата Мура.
Результатом применения алгоритма стало уменьшение числа состояний с сохранением
реакции на входное слово.

\end{document}
