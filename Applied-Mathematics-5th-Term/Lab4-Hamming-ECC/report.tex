\documentclass[listings]{labreport}
\usepackage{csvsimple}
\usepackage{verbatim}
\subject{Прикладная математика}
\titleparts{Лабораторная работа №4}{Построение кода Хэмминга}
\students{Лабушев Тимофей}

\makeatletter
\patchcmd{\verbatim@input}{\@verbatim}{\footnotesize\@verbatim}{}{}
\makeatother

\begin{document}

\maketitlepage

\section*{Цель работы}

Получить практические навыки кодирования и декодирования
помехоустойчивыми кодами Хэмминга.

\section*{Задание}

\begin{enumerate}
\item Реализовать процедуру построения кода Хэмминга с заданным
  числом информационных символов с кодовыми расстояниями 3 и 4.
\item Реализовать процедуру декодирования кода Хэмминга и
  исправления ошибок для различных кодовых расстояний. Провести
  программный контроль выполнения на примере случайных кодовых
  комбинаций.
\end{enumerate}

\section*{Исходный код}

\subsection*{Кодирование}

\verbatiminput{hamming.rkt}

\subsection*{Тестирование}

\verbatiminput{hamming-test.rkt}

\section*{Выводы}

В ходе выполнения лабораторной работы были изучены приципы построения
помехоустойчивых кодов Хэмминга для исправления одной ошибки (SEC) и
исправления одной с обнаружением двух (SECDED).

\end{document}
