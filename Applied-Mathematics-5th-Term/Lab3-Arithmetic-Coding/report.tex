\documentclass[listings]{labreport}
\usepackage{csvsimple}
\usepackage{verbatim}
\subject{Прикладная математика}
\titleparts{Лабораторная работа №3}{Арифметическое кодирование}
\students{Лабушев Тимофей}

\makeatletter
\patchcmd{\verbatim@input}{\@verbatim}{\footnotesize\@verbatim}{}{}
\makeatother

\begin{document}

\maketitlepage

\section*{Цель работы}

Получить практические навыки кодирования и декодирования текстового файла
арифметическим кодом.

\section*{Задание}

\begin{enumerate}
\item Реализовать процедуру построения арифметического кода
\item Предусмотреть ввод последовательности символов для кодирования
\item Вычислить коэффициент сжатия данных как процентное отношение
  длины закодированного файла к длине исходного файла
\item Декодировать полученную последовательность,
  сравнить полученный файл с исходным текстом
\end{enumerate}

\section*{Исходный код}

\subsection*{Кодирование}

\verbatiminput{coder.rkt}

\subsection*{Ввод}

\verbatiminput{entry.rkt}

\section*{Пример работы программы}

\begin{scriptsize}
\begin{verbatim}
Enter the text to encode:
Fly me to the moon, let me play among the stars, and let me see what spring is like on Jupyter and Mars
Enter desired precision (in bits, 53 for double precision)
450
Encoded: 0.9821783982146021614208059043513765417781895896223288305328811410486010602429103146656947712082555708095342416578377510963963423517698176
Decoded: Fly me to the moon, let me play among the stars, and let me see what spring is like on Jupyter and Mars
Encoding precision: 103
Compression ratio: 54.61165048543689%
\end{verbatim}
\end{scriptsize}

\section*{Выводы}

В ходе выполнения лабораторной работы было установлено, что для кодирования
сообщений достаточной длины арифметическим кодированием необходима высокая точность
представления чисел с плавающей запятой.

\end{document}
