\documentclass[listings]{labreport}
\usepackage{csvsimple}
\usepackage{verbatim}
\subject{Прикладная математика}
\titleparts{Лабораторная работа №5}{Циклическое кодирование}
\students{Лабушев Тимофей}

\makeatletter
\patchcmd{\verbatim@input}{\@verbatim}{\footnotesize\@verbatim}{}{}
\makeatother

\begin{document}

\maketitlepage

\section*{Цель работы}

Получить практические навыки построения циклического кода
по заданным характеристикам и проверка его свойства
по обнаружению и исправлению ошибок.

\section*{Задание}

\begin{enumerate}
\item Вычислить параметры кода и найти образующий многочлен,
  воспользовавшись таблицей неприводимых многочленов. 
\item Проверить, имеются ли ошибки в исследуемой комбинации,
  при наличии ошибок – исправить их.
\item Провести программный контроль выполнения
  на примере случайных кодовых комбинаций.
\end{enumerate}

\section*{Исходный код}

\subsection*{Кодирование}

\verbatiminput{coder.rkt}

\subsection*{Битовые операции}

\verbatiminput{bit-utils.rkt}

\subsection*{Тестирование}

\verbatiminput{coder-test.rkt}

\section*{Выводы}

В ходе выполнения лабораторной работы были изучены приципы построения
и декодирования циклических кодов.

\end{document}
