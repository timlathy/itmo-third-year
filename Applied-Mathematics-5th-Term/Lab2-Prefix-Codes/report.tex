\documentclass[listings]{labreport}
\usepackage{csvsimple}
\usepackage{verbatim}
\subject{Прикладная математика}
\titleparts{Лабораторная работа №2}{Построение оптимальных кодов}
\students{Лабушев Тимофей}

\makeatletter
\patchcmd{\verbatim@input}{\@verbatim}{\footnotesize\@verbatim}{}{}
\makeatother

\begin{document}

\maketitlepage

\section*{Цель работы}

Изучение основных принципов эффективного кодирования и приобретение
практических навыков построения оптимальных кодов на примере
кодов Шеннона-Фано и Хаффмана, оценка их эффективности.

\section*{Задание}

\begin{enumerate}
\item Реализовать процедуры построения кода Шеннона-Фано и оптимального кода Хаффмана 
\item Построить коды для текстового файла, распечатать кодовые таблицы, содержащие
\textit{символ}, \textit{вероятность}, \textit{кодовое слово},
\textit{длину кодового слова}.
\item Сравнить среднюю длину кодовых слов, полученных двумя алгоритмами.
\end{enumerate}

\section*{Исходный код}

\subsection*{Построение кода Хаффмана}
\verbatiminput{huffman.rkt}

\subsection*{Построение кода Шеннона-Фано}
\verbatiminput{shannon-fano.rkt}

\section*{Результаты работы программы}

\csvstyle{codelist}{tabular=|c|c|c|c|,
  table head=\hline Символ & Вероятность & Кодовое слово & Длина кодового слова\\\hline,
  late after line=\\\hline}

\subsection*{Код Хаффмана}

\noindent
\csvreader[codelist]{huffman.csv}{}{\csvcoli & \csvcolii & \csvcoliii & \csvcoliv}

\subsection*{Код Шеннона-Фано}

\noindent
\csvreader[codelist]{shannon-fano.csv}{}{\csvcoli & \csvcolii & \csvcoliii & \csvcoliv}

\section*{Сравнение кодов}

Найдем среднюю длину кодового слова, взвешенную по вероятности появления в тексте:\\

Код Хаффмана:
\verbatiminput{huffman.stat}

Код Шеннона-Фано:
\verbatiminput{shannon-fano.stat}

\section*{Выводы}

В ходе выполнения лабораторной работы было установлено, что код Хаффмана
является более оптимальным с учетом вероятности появления символов в исходном тексте.

\end{document}
