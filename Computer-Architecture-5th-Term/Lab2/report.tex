\documentclass[ru, listings]{labreport}
\subject{Организация ЭВМ и систем}
\titleparts{Лабораторная работа №2}{Ввод-вывод численных данных}
\students{Нестеров Дали \\[1mm] Лабушев Тимофей}

\begin{document}

\maketitlepage

\section*{Цель работы}

Познакомиться с двоично-десятичным и двоичным представлением целых и дробных чисел.

Совместить перевод из 10 в 2 и из 2 в 10 в одной программе для целых и дробных чисел
и разработать программы на С51 и в Ассемблере A51 для ввода и вывода двузначных чисел.
Сравнить листинги .lst программ в С51 и А51 и пояснить различия в программах. 

\section*{Исходный текст программы на C51}

\lstinputlisting[language=c, basicstyle=\scriptsize]{program.c}

\section*{Исходный текст программы на A51}

\lstinputlisting[language=c, basicstyle=\scriptsize]{program.asm}

\section*{Сравнение листингов}

\subsection*{Размер кода}

C51: \verb|Program Size: data=9.0 xdata=0 code=290|

A51: \verb|Program Size: data=8.0 xdata=0 code=152|

\subsection*{Пояснение}

В отличии от скомпилированного кода, написанный вручную ассемблерный код
более эффективно использует регистры и выполняет операции. Например,
код на C выполняет операцию деления три раза для следующей конструкции:

\begin{verbatim}
bin = (bin % 100 > 50) ? bin / 100 + 1 : bin / 100;
\end{verbatim}

В то время как ассемблерный код использует результат (частное и остаток) одной операции.
Это отражается не только на размере кода, но и на скорости выполнения, поскольку
операция 16-битного деления не входит в набор команд 8051 и реализуется программно.

\section*{Вывод}

\end{document}

