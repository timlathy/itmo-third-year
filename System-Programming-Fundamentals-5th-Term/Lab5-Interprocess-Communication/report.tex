\documentclass[en, listings]{labreport}
\subject{System Software Fundamentals}
\titleparts{Lab Work \#5}{Interprocess Communication and Thread Synchronization}
\students{Timothy Labushev}

\lstdefinestyle{lablisting}{
  basicstyle=\scriptsize\ttfamily,
  numbers=left,
  stepnumber=1,
  numbersep=10pt,
  showspaces=false,
  showstringspaces=false
}

\usepackage{verbatim}

\begin{document}

\maketitlepage

\section*{Assignment}

\subsection*{Part I}

Using C, implement a server that stores its \verb|PID|, \verb|UID|, \verb|GID| on launch
and continuously updates:
\begin{itemize}
\item The number of seconds passed since the process start;
\item Average system load for the last 1, 5, and 15 minutes.
\end{itemize}

The server should support the following communication channels:
\begin{enumerate}
\item System V shared memory
\item System V message queue
\item Memory-mapped file
\end{enumerate}

\subsection*{Part II}

Write a progam in C which creates an array of lowercase Latin letters and
mutates it in concurrently running threads. \textit{Thread \#1} inverts the case,
\textit{thread \#2} reverses the array in place.

The following operations should be implemented:
\begin{enumerate}
\item Each second the main thread alternately wakes up one of the mutating threads,
  waits until they complete an operation, and outputs the array to \verb|stdout|.
  \textit{POSIX semaphores} are used for interthread synchronization.
\item Each second the main thread alternately wakes up one of the mutating threads,
  waits until they complete an operation, and outputs the array to \verb|stdout|.
  \textit{System V semaphores} are used for interthread synchronization.
\item The main thread outputs the array to \verb|stdout| at a fixed time interval
  (in microseconds). The threads perform their operations at a fixed time interval
  (in microseconds) as well, locking the shared resource using
  \textit{pthread mutexes}.
\item The main thread outputs the array to \verb|stdout| at a fixed time interval
  (in microseconds). \textit{Thread \#1} and \textit{thread \#2} perform their
  operations at a fixed time interval (in microseconds) as well,
  acquiring the shared resource for writing. Yet another thread prints the
  number of uppercase letters in the array at a fixed time interval,
  acquiring the shared resource for reading. \textit{pthread read-write locks}
  are used for interthread synchronization.
\end{enumerate}

\subsection*{Part III}

The following tasks should be done in C and Perl:
\begin{enumerate}
\item Implement a client-server pair communicating over Unix domain sockets.
  The behavior of the server should be based on the first part of the assignment.
\item Add custom signal handling for \verb|HUP|, \verb|INT|, \verb|TERM|,
  \verb|USR1|, \verb|USR2| on the server.
\item Create a program that launches a child process via \verb|fork()| and
  replaces its \verb|stdin| with the read end of an unnamed pipe. The parent
  process outputs each second character of a file specified in arguments
  to the write end of the pipe. The child process launches the \textit{wc}
  utility counting characters.
\end{enumerate}

\section*{General Requirements}

\begin{itemize}
\item Write a Makefile to compile the source code to separate executables,
  one for each assignment
\item Insert \verb|use strict; use warnings qw(FATAL all);| in Perl scripts
  and enable \textit{taint mode} with \verb|#!/usr/bin/perl -T|.
\end{itemize}

\section*{Code Listing}

\verb|.c| and \verb|.pl| files are available at
\begin{verbatim}
https://github.com/timlathy/itmo-third-year/tree/master/
System-Programming-Fundamentals-5th-Term/Lab5-Interprocess-Communication
\end{verbatim}

\subsection*{Makefile}

\lstinputlisting[style=lablisting]{Makefile}

\end{document}
