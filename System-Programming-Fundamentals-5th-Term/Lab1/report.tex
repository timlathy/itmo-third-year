\documentclass[en, listings]{labreport}
\subject{System Software Fundamentals}
\titleparts{Lab Work \#1 (9)}{Introduction to Shell Scripting}
\students{Timothy Labushev}

\begin{document}

\maketitlepage

\section*{Assignment}

This report discusses the implementation of an interactive shell script that
performs a fixed set of actions: \textit{print the current working directory},
\textit{list files in the current working directory}, \textit{create a new directory},
\textit{grant and revoke write access to a directory}.

For commands that require a directory name to be provided, the user may use
TAB completion and \texttt{vi} or \texttt{emacs} keybindings. This is achieved by invoking
the \texttt{read} utility with an \verb|-e| switch, which uses GNU Readline to obtain a line
from standard input. It is important to note that it is not mandated by POSIX and may be
absent on uncommon and outdated systems.

Error handling is performed by checking the exit code of executed commands.
In case of a non-zero code, a user-friendly error message printed. The error message
as produced by the failed command is appened to a log file, which can be inspected later.

\section*{Code Listing}

\lstinputlisting[firstline=3, basicstyle=\scriptsize]{lab.sh}

\end{document}
