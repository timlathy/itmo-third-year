\documentclass[en, listings]{labreport}
\subject{System Software Fundamentals}
\titleparts{Lab Work \#4 (3)}{System Calls}
\students{Timothy Labushev}

\lstdefinestyle{lablisting}{
  basicstyle=\scriptsize\ttfamily,
  numbers=left,
  stepnumber=1,
  otherkeywords={EOF, O\_RDONLY, STDIN\_FILENO, STDOUT\_FILENO, STDERR\_FILENO},
  numbersep=10pt,
  showspaces=false,
  showstringspaces=false
}

\usepackage{verbatim}

\begin{document}

\maketitlepage

\section*{Assignment}

\subsection*{Part I}

Using raw system calls, write a C program that mimics the behavior of the \textit{head} utility.

\subsubsection*{Requirements}

\begin{enumerate}
\item The program should perform input and output via \verb|read(2)| and \verb|write(2)|.
\item The program should accept multiple input files and use standard input when \texttt{-} is given as a filename.
\item The program should handle errors and print informative messages to \verb|stderr|.
\end{enumerate}

\subsection*{Part II}

Rewrite the same program in Perl.

\subsubsection*{Requirements}

\begin{enumerate}
\item Use \verb|use strict; use warnings qw(FATAL all);| pragmas.
\item Enable \textit{taint mode} with \verb|#!/usr/bin/perl -T|.
\end{enumerate}

\subsection*{Part III}

Write a C program that mimics the behavior of the \textit{xargs}
utility without optional arguments.

\subsection*{Relevant Sections from POSIX.1-2008}

\subsubsection*{Synopsis}

\noindent
\texttt{head [-n \textit{number}] [\textit{file}...]}

\subsubsection*{Description}

\noindent
The \textit{head} utility shall copy its input files to the standard output,
ending the output for each file at a designated point.

\noindent
Copying shall end at the point in each input file indicated by the
\textbf{-n} \textit{number} option. The option-argument number shall be counted in units of lines.

\subsubsection*{Options}

\noindent
The following option shall be supported:

\textbf{-n} \textit{number}

The first \textit{number} lines of each input file shall be copied to standard output.
The application shall ensure that the number option-argument is a positive decimal integer.

\noindent
When a file contains less than number lines, it shall be copied to standard output
in its entirety. This shall not be an error.

\noindent
If no options are specified, \textit{head} shall act as if \textbf{-n 10} had been specified.

\subsubsection*{Stdout}

\noindent
The standard output shall contain designated portions of the input files.

\noindent
If multiple \textit{file} operands are specified, \textit{head} shall precede
the output for each with the header:

\begin{verbatim}
"\n==> %s <==\n", <pathname>
\end{verbatim}

\noindent
except that the first header written shall not include the initial <newline>.

\section*{Code Listing}

\subsection*{head (C)}

\setmonofont{Fira Mono}
\lstinputlisting[language=c, style=lablisting]{head.c}

\subsection*{head (Perl)}

\lstinputlisting[language=perl, style=lablisting]{head.pl}

\subsection*{xargs (C)}

\lstinputlisting[language=c, style=lablisting]{xargs.c}

\end{document}
