\documentclass[listings]{labreport}
\usepackage{amsmath}
\subject{Методы оптимизации}
\titleparts{Расчетная работа №1}{Вариант 8}
\students{Лабушев Тимофей}

\begin{document}

\maketitlepage

\section*{Симплекс-метод}

$$f = 0.1x_1 + 0.075x_2 \to min$$

Приведем задачу к каноническому виду, введя дополнительные переменные $x_3, x_4, x_5$:

$$
\begin{cases}
  x_2 - x_3 = 3 \\
  3x_1 + 2.67x_2 - x_4 = 27 \\
  x_1 + 2.27x_2 - x_5 = 16 \\
  x_i \geqslant 0,\ i = \overline{1,5}
\end{cases}
$$

Применим метод искусственного базиса. Для этого введем переменные $y_1, y_2, y_3$:

$$
\begin{cases}
  x_2 - x_3 + y_1 = 3 \\
  3x_1 + 2.67x_2 - x_4 + y_2 = 27 \\
  x_1 + 2.27x_2 - x_5 + y_3 = 16 \\
  x_i \geqslant 0,\ i = \overline{1,5} \\
  y_j \geqslant 0,\ j = \overline{1,3}
\end{cases}
$$

Будем решать вспомогательную задачу $W = y_1 + y_2 + y_3 \to min$.

$$y_1 = -x_2 + x_3 + 3$$
$$y_2 = -3 x_1 - 2.67 x_2 + x_4 + 27$$
$$y_3 = -x_1 - 2.27 x_2 + x_5 + 16$$
$$W = -4 x_1 - 5.94 x_2 + x_3 + x_4 + x_5 + 46$$

\begin{center}
\begin{tabular}{c|cccccc}
& $x_1$ & $x_2$ & $x_3$ & $x_4$ & $x_5$ & $\delta$ \\\hline
$y_1$ & 0 & -1 & 1 & 0 & 0 & 3 \\
$y_2$ & -3 & -2.67 & 0 & 1 & 0 & 27 \\
$y_3$ & -1 & -2.27 & 0 & 0 & 1 & 16 \\\hline
$W$ & -4 & -5.94 & 1 & 1 & 1 & 46 \\
\end{tabular}
\end{center}

Выберем наибольшую по модулю отрицательную $\delta$: $x_2$. При увеличении $x_2$
быстрее всего до нуля доходит $y_1$ — поменяем их местами.

$$x_2 = -y_1 + x_3 + 3$$
$$y_2 = -3 x_1 - 2.67 x_3 + x_4 + 2.67 y_1 + 18.99$$
$$y_3 = -x_1 - 2.27 x_3 + x_5 + 2.27 y_1 + 9.19$$
$$W = -4 x_1 - 4.94 x_3 + x_4 + x_5 + 5.94 y_1 + 28.18$$

\begin{center}
\begin{tabular}{c|cccccc}
& $x_1$ & $y_1$ & $x_3$ & $x_4$ & $x_5$ & $\delta$ \\\hline
$x_2$ & 0 & -1 & 1 & 0 & 0 & 3 \\
$y_2$ & -3 & 2.67 & -2.67 & 1 & 0 & 18.99\\
$y_3$ & -1 & 2.27 & -2.27 & 0 & 1 & 9.19\\\hline
$W$ & -4 & 5.94 & -4.94 & 1 & 1 & 28.18\\
\end{tabular}
\end{center}

Выберем наибольшую по модулю отрицательную $\delta$: $x_3$. При увеличении $x_3$
быстрее всего до нуля доходит $y_3$ — поменяем их местами.

$$x_3 = -0.44 x_1 + 0.44 x_5 + y_1 - 0.44 y_3 + 4.05$$
$$x_2 = -0.44 x_1 + 0.44 x_5 - 0.44 y_3 + 7.05$$
$$y_2 = -1.8252 x_1 + x_4 - 1.1748 x_5 + 1.1748 y_3 + 8.1765$$
$$W = -1.8264 x_1 + x_4 - 1.1736 x_5 + y_1 + 2.1736 y_3 + 8.173$$

\begin{center}
\begin{tabular}{c|cccccc}
& $x_1$ & $y_1$ & $y_3$ & $x_4$ & $x_5$ & $\delta$ \\\hline
$x_2$ & -0.44 & 0 & -0.44 & 0 & 0.44 & 7.05 \\
$y_2$ & -1.8252 & 0 & 1.1748 & 1 & -1.1748 & 8.1765\\
$x_3$ & -0.44 & 1 & -0.44 & 0 & 0.44 & 4.05\\\hline
$W$ & -1.8264 & 1 & 2.1736 & 1 & -1.1736 & 8.173\\
\end{tabular}
\end{center}

Выберем наибольшую по модулю отрицательную $\delta$: $x_1$. При увеличении $x_1$
быстрее всего до нуля доходит $y_2$ — поменяем их местами.

$$x_1 = 0.548 x_4 - 0.644 x_5 - 0.548 y_2 + 0.644 y_3 + 4.48$$
$$x_3 = -0.241 x_4 + 0.723 x_5 + y_1 + 0.241 y_2 - 0.723 y_3 + 2.079$$
$$x_2 = -0.241 x_4 + 0.723 x_5 + 0.241 y_2 - 0.723 y_3 + 5.079$$
$$W = y_1 + y_2 + y_3$$

\begin{center}
\begin{tabular}{c|cccccc}
& $y_2$ & $y_1$ & $y_3$ & $x_4$ & $x_5$ & $\delta$ \\\hline
$x_2$ & 0.241 & 0 & -0.723 & -0.241& 0.723 & 5.079 \\
$x_1$ & -0.548 & 0 & 0.644 & 0.548 & -0.644 & 4.48 \\
$x_3$ & 0.241 & 1 & -0.724 & -0.241 & 0.723 & 2.079\\\hline
$W$ & 1 & 1 & 1 & 0 & 0 & 0\\
\end{tabular}
\end{center}

Критерий оптимальности выполнен: все $\delta \geqslant 0$. Вспомогательная
задача решена. Вернемся к исходной задаче, удалив вспомогательные переменные
$y_1$, $y_2$, $y_3$:

$$x_1 = 0.548 x_4 - 0.644 x_5 + 4.48$$
$$x_2 = -0.241 x_4 + 0.723 x_5 + 5.079$$
$$x_3 = -0.241 x_4 + 0.723 x_5 + 2.079$$
$$f = 0.1x_1 + 0.075x_2 = 0.0367 x_4 - 0.0102 x_5 + 0.8289$$

\begin{center}
\begin{tabular}{c|ccc}
& $x_4$ & $x_5$ & $\delta$ \\\hline
$x_1$ & 0.548 & -0.644 & 4.48 \\
$x_2$ & -0.241 & 0.723 & 5.079 \\
$x_3$ & -0.241 & 0.723 & 2.079 \\
$-f$ & -0.0367 & 0.0102 & -0.8289 \\\hline
\end{tabular}
\end{center}

Полученное решение неоптимально: коэффициенты при свободных переменных в индексной
cтроке отрицательные. Для получения опорного плана перейдем к новой таблице.
Выберем наибольшую положительную $\delta$: $x_5$. При увеличении $x_5$
быстрее всего до нуля доходит $x_1$ — поменяем их местами.

$$x_5 = -1.553 x_1 + 0.85 x_4 + 6.956$$
$$x_2 = -1.123 x_1 + 0.374 x_4 + 10.108$$
$$x_3 = -1.123 x_1 + 0.374 x_4 + 7.108$$
$$f = 0.0158 x_1 + 0.028 x_4 + 0.758$$

\begin{center}
\begin{tabular}{c|ccc}
& $x_4$ & $x_1$ & $\delta$ \\\hline
$x_5$ & 0.85 & -1.553 & 6.956 \\
$x_2$ & 0.374 & -1.123 & 10.108 \\
$x_3$ & 0.374 & -1.123 & 7.108 \\
$-f$ & -0.028 & -0.0158 & -0.758 \\\hline
\end{tabular}
\end{center}

Оптимальное решение найдено:

$$x_1* = 0,\ x_2* = 10.108,\ f* \approx 0.758$$

\end{document}
